\documentclass[12pt,a4paper]{scrartcl}
\usepackage{UoYstyle}
\title{A \LaTeX\ package to provide\\The University of York's house style}
\author{Jeremy Jacob}
\date{2011 January 24}
\hypersetup{
  pdfauthor={Jeremy L. Jacob},
  pdftitle={A LaTeX package to provide The University of York's house style},
  pdfsubject={A LaTeX style for University of York documents}
  }
\DeclareRobustCommand{\cmd}[2]{\textsf{\textbackslash#1}\textit{#2}}
\DeclareRobustCommand{\logo}[3]{%
  \begin{center}
    \begin{tabular}[b]{l}
      \cmd{#1}{[height]}\\#2\\#3
    \end{tabular}
    \hfill
    \fbox{\csname #1\endcsname}
  \end{center}
}

\begin{document}
\maketitle

\section{Introduction}
\label{sec:inroduction}

The \LaTeXe\ package \textsf{UoYstyle} sets up various things to
provide the standard style for University of York documents in
(British) English (\autoref{sec:packages}).  It loads the Palatino
fonts, together with automatic margin kerning
(\autoref{sec:packages}), defines names for the University's colours
(\autoref{sec:colour}) and defines commands to typeset various
logotypes (\autoref{sec:logotype}): University logotypes, University
woodcuts and the Department of Computer Science logotype and

Each logotype is obtained by a command which has an optional
parameter: the height of the logotype.  Each logotype, except for the
three in the \cmd{UoY$*$logo}{} family, is positioned at whatever
position the source defines.  Each of the \cmd{UoY$*$logo}{} family is
lowered slightly, to look as though it sits on the same line as the
surrounding text.

\section{Package options and packages loaded}
\label{sec:packages}

There is one options: \emph{logos} (default) or \emph{nologos}.  In
the event of logotypes not being available they can be turned off by
giving the \emph{nologos} option.

\textsf{UoYstyle} loads the following packages:
\begin{itemize}
\item \textsf{babel} (with option \textsf{british)};
\item \textsf{mathpazo} (the university font for text and
  mathematics);
\item \textsf{microtype}
\item \textsf{graphics} (to load the
  graphics files for the logotypes);
\item \textsf{color} (to define
  University and Departmental colours); and
\item \textsf{hyperref}.
\end{itemize}

The package determines the output format being used (either DVI from
\textsf{latex} for eventual conversion to PostScript by
\textsf{dvips}, or PDF from \textsf{pdflatex}) to choose an output
driver for the sub-packages requiring one.

The package looks for the logotype files in
\begin{itemize}
\item \textsf{/sys/lib/postscript/logo},
\item \textsf{/sys/lib/pdf/logo},
\item \textsf{/usr/local/share/postscript/logo} or
\item \textsf{/usr/local/share/pdf/logo}
\end{itemize}
as appropriate.  If an appropriate one of these directories does not
exist the package will fail catastrophically.

Other options for \textsf{hyperref} must be set up using
\cmd{hypersetup}{$\{$\ldots$\}$}.

\section{Defined colours}
\label{sec:colour}
\newcommand*{\showcol}[1]{\textsf{#1}~\textcolor{#1}{\rule{1em}{1em}}}
Several `official' colours are defined:

For the University colour scheme: \showcol{UoYblue} and
\showcol{UoYgreen}.

For the Departmental colour scheme: \showcol{UoYCSbrightgreen},
\showcol{UoYCSyellowgreen}, \showcol{UoYCSgrey} and
\showcol{UoYCSblue}.

These can be used in the normal way as the first argument of a colour
macro such as: \cmd{color}{}, \cmd{textcolor}{}, \cmd{colorbox}{},
\cmd{fcolorbox}{} and \cmd{pagecolor}{}.

\section{Commands to produce the logotypes}
\label{sec:logotype}

Each logotype is described with the command that produces it, its
default height and the name of the file in the default directory that
produces it.  The border around each logotype is merely to show the
logotype's extent; the border is not part of the logotype.

If the \emph{nologos} option is given then a warning message is
printed.

The directory containing the logotype file must be in your \TeX\
search path.  On centrally maintained Departmental linux systems, with
processing by \textsf{pdflatex}, you can do this by setting:\\
\textsf{TEXINPUTS=\$TEXINPUTS:/usr/share/local/pdf/logo:/usr/local/share/THE-2010}

\subsection{University}

\logo{UoYbwlogo}{3.5ex}{uoy}

\logo{UoYbluelogo}{3.5ex}{uoy-p281}

\logo{UoYgreenlogo}{3.5ex}{uoy-p567}

\logo{Woodcut}{7ex}{univwoodcut}

\logo{UoYbluewoodcut}{7ex}{wct-p281}

\logo{UoYgreenwoodcut}{7ex}{wct-p567}

\subsection{Department}

The old departmental logotypes have been withdrawn, and are no longer
supported.

\subsection{Times Higher Education University of the Year Award}

Support for these will be withdrawn sometime after the award lapses.

\logo{THESrgb}{10ex}{Logo\_large\_RGB}

\logo{THESgrey}{10ex}{Logo\_large\_grey}

\end{document}
