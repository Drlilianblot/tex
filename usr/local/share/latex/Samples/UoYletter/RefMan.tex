\documentclass[12pt,a4paper]{article}
\usepackage{UoYstyle,harvard}
\newcommand{\cmdsty}[1]{\textsf{#1}}
\newcommand{\cmd}[1]{\cmdsty{\textbackslash#1}}
\newcommand{\pcmd}[2]{\cmdsty{\cmd{#1}$\{$#2$\}$}}
\newcommand{\nl}{\textbackslash\textbackslash}
\title{Typesetting University of York Letters:\\
                \LaTeX\ class \cmdsty{UoYletter}}
\author{Jeremy Jacob
        \thanks{Department of Computer Science;
                Ext 2747; $<$jeremy@minster.york.ac.uk$>$}}
\date{10 September 2010}
\begin{document}
\maketitle
\section{Introduction}
\LaTeX\ class \cmdsty{UoYletter} is provided for typesetting
University of York letters.  It produces letters which are as close to
the University's ``authorised'' style (including forcing fields into
the correct typeface) as I can make it, given the information
provided.

\cmdsty{UoYletter} inherits from the standard \cmdsty{letter} style:
see \citeasnoun{Lamport1994} for a description.  The way letters are
typed with \cmdsty{UoYletter} is similar to that of \cmdsty{letter}:
the environment \cmdsty{letter} delimits a letter, and takes a
parameter containing the ``To:'' address; commands
\pcmd{opening}{\ldots} and \pcmd{closing}{\ldots} are required; and
\pcmd{encl}{\ldots}, \pcmd{ps}{\ldots} are available.
 
\cmdsty{UoYletter} does two things for you:
\begin{itemize}
\item it replaces the \cmd{address} declaration by a collection of
declarations (see Section \ref{NewDec:Sec}); and
\item it provides a few extra commands for use within your document
(see Section \ref{NewCom:Sec}).
\end{itemize}

For more general information about \LaTeX\ see
\citeasnoun{Lamport1994} and the ``Local Guide''.

\section{Class options}\label{ClOpts:Sec}
The options to \cmdsty{UoYletter} are identical to those of
\cmdsty{letter}, except for paper size, which is fixed at
\cmdsty{a4paper} (again, the reader is referred
\citeasnoun{Lamport1994}) and for options to control the printing of
the woodcut.  The extra options come in two groups, one to choose
monochrome or polychrome and one to choose the logotype.  They are:
\begin{description}
\item[\cmdsty{monochrome}] all logotypes and text in black-and-white
  (this is the default); and
\item[\cmdsty{polychrome}] colour logotypes and the opportunity change
  the colour of the headings from black (see
  Section~\ref{NewDec:Sec}).
\end{description}
and
\begin{description}
\item[\cmdsty{qapfhe}] to get the Queen's Anniversary Prize logotype
  (this should not be used except to print off hard copies of old
  letters).
\item[\cmdsty{woodcut}] to get the University Woodcut (this is the
  default); and
\item[\cmdsty{nowoodcut}] to get neither.
\end{description}
If more than one of these options are given the latest, as given by
this list, is used.

\section{New declarations}\label{NewDec:Sec}

The declarations come in four kinds:
\begin{itemize}
\item for the author's group (see Subsection~\ref{GrpDec:SS});
\item for the author's personal details (see
  Subsection~\ref{PerDec:SS});
\item letter specific (see Subsection~\ref{LetDec:SS}); and
\item colour settings (see Subsection~\ref{Colour:SS}).
\end{itemize}
\subsection{Group declarations}\label{GrpDec:SS}
Group information consists of a mandatory group name (for example
``Department of History'', ``Printing Unit'' or ``High Integrity
Software Engineering Group'') and some optional fields.
If the group is a department, the department's name is declared with
\pcmd{department}{name}.  For example, \pcmd{department}{History}.
If the group is neither a department nor a college, its name should be
declared using \pcmd{groupname}{name}.  The \cmdsty{name} may include
newline commands (\nl) and emboldening commands
(\pcmd{textbf}{\ldots}); \pcmd{department}{name} is equivalent to:\\
\pcmd{groupname}{Department of\ \nl \pcmd{textbf}{name}}.


The optional declarations are:
\begin{itemize}
\item campus: either \cmd{HesEast} or \cmd{HesWest} (default is
  \cmd{HesEast});
\item title of the head of the group (\pcmd{groupheadtitle}{title});
\item name of the head of the group (\pcmd{grouphead}{name});
\item group telephone extension (\pcmd{grouptelext}{dddd});
\item group facsimile extension (\pcmd{groupfaxext}{dddd}); and
\item electronic mail domain (\pcmd{emaildomain}{full.domain.name}).
\end{itemize}

For example, for the Department of Computer Science the following
declarations might be made:\\
\hspace*{1cm}\begin{tabular}[t]{l}
\pcmd{department}{Computer Science}\\
\pcmd{grouphead}{Ian Pyle}\\
\pcmd{groupheadtitle}{Head of Department}\\
\pcmd{grouptelext}{2722}\\
\pcmd{groupfaxext}{2767}
\end{tabular}\\
However, in the case of The Department of Computer Science, a single
declaration, \cmd{ComputerScience}, is provided; it is equivalent to
the preceeding five declarations, but with up-to-date values!

\subsection{Personal declarations}\label{PerDec:SS}

There is one mandatory declaration, \pcmd{signature}{name}, exactly as
in class \cmdsty{letter}.

The new personal declarations are all optional.  They are used to give
\begin{itemize}
\item a formal variant of your name (\pcmd{namequal}{name and
qualifications}), this is different from your signature;
\item a job title (e.g. ``Lecturer'', ``Research Fellow'')
(\pcmd{jobtitle}{string}); and
\item your login name for electronic mail (\pcmd{loginid}{string}).
\end{itemize}

\subsection{Letter specific declarations}\label{LetDec:SS}

The letter specific declarations include, from \cmdsty{letter},
\pcmd{date}{string}, with default \cmd{today}.

In addition, there are declarations for including an optional:
\begin{itemize}
\item ``your reference'' field (\pcmd{yourref}{string}); and
\item ``our reference'' field (\pcmd{ourref}{string}). 
\end{itemize}

The declarations \cmd{national} and \cmd{international} change the
headings for inland and overseas letters respectively; the default is
\cmd{national}.

\subsection{Colour setting commands}\label{Colour:SS}

To use colour within a \cmdsty{polychrome} letter body just use the
standard \LaTeX\ colour commands (see Section~7.3 of
\citeasnoun{Lamport1994}).  Within each heading declaration colour can
be set on a character by character basis if you wish.  However a
colour for the whole header (excluding the ``to address'') can be
given with the \pcmd{headcolour}{colour} declaration.  The default
colour is \cmdsty{black}.  In a \cmdsty{monochrome} document this
declaration is ignored, but any colour changes you program by hand
will work (or not) as described in the \LaTeX\ book.

\section{New commands}\label{NewCom:Sec}

The two commands \pcmd{about}{string} and \pcmd{re}{string} may be
used in the body of the letter (traditionally immediately after the
greeting) to give the subject of the letter, preceeded by
``\textbf{Re:}'' in the latter case.  The \cmdsty{string} may contain
newline commands (\nl).

\section{Example}

An example of an input file is given in Figure \ref{ExampleInput:Fig}.
\begin{figure}
{\sffamily
\begin{tabular}[t]{l}
\pcmd{documentclass[12pt]}{UoYletter}\\[1ex]
\cmd{ComputerScience}\\
\pcmd{signature}{Hes Lington-Hall}\\
\pcmd{namequal}{Prof. H. Lington-Hall, CDM, DDT}\\
\pcmd{jobtitle}{Layabout}\\
\pcmd{extension}{6666}\\
\pcmd{loginid}{hlh}\\
\pcmd{emaildomain}{devnull.york.ac.uk}
        \% must follow ``\cmd{ComputerScience}''\\
\pcmd{website}{http://www.java.applets.R.us/} \% ditto\\[1ex]
\pcmd{definecolor}{brown$\}\{$rgb$\}\{$0.7,0.5,0.3}\\
\pcmd{headcolour}{brown}\\[1ex]
\pcmd{date}{31 February 1944}\\[1ex]
\pcmd{begin}{document}\\
\pcmd{begin}{letter}\\
\hspace*{1cm}   $\{$Dr. A. Cademic\nl The House\nl Avenue Road\nl
                The City\nl CI9 0OO$\}$\\
\pcmd{ourref}{Sp/1/H-LH/GS}\\
\pcmd{yourref}{23 7-A/b}\\
\pcmd{opening}{Dear Arbuthnot,}\\
\pcmd{re}{Arachnophilia}\\
Thank you for your letter of 5th \pcmd{emph}{inst}.\\
The Tarantulas finished off Agatha.\\
I hope you enjoy the enclosed;\\
perhaps it could help with the\\
Problem Of The One Undergraduate Too Many!\\
\pcmd{closing}{Lurv `n' kisses,}\\
\pcmd{encl}{A Black Widow}\\
\pcmd{ps}{PS The Bird Eating Spiders don't.}\\
\pcmd{end}{letter}\\
\pcmd{end}{document}
\end{tabular}
}
\caption{An example input file}\label{ExampleInput:Fig}
\end{figure}
It shows how default values in a declaration such as
\cmd{ComputerScience} can be overridden as well as where the
declarations may be placed.  This file (\cmdsty{Example.tex}) and its
output (\cmdsty{Example.dvi}) may be found in the same directory as
the class file \cmdsty{UoYletter}.

\section{Making a personal letter class}

It is possible to save typing by creating your own sub-class of
\cmdsty{UoYletter}.  By creating a file called \cmdsty{HLHletter.cls}
with contents as in Figure \ref{Cls:Fig} the example in Figure
\ref{ExampleInput:Fig} could have been typed as in Figure
\ref{ExampleInpTwo:Fig}.
\begin{figure}
{\sffamily
\begin{tabular}[t]{l}
\pcmd{NeedsTeXFormat}{LaTeX2e}\\
\pcmd{ProvidesClass}{HLHletter}[1994/09/13 Private Class for HLH]\\
\pcmd{DeclareOption*}
        {\pcmd{PassOptionsToClass}{\cmd{CurrentOption}}$\{$UoYLetter$\}$}\\
\cmd{ProcessOptions}\\
\pcmd{LoadClass}{UoYletter}\\
\cmd{ComputerScience}\\
\pcmd{signature}{Hes Lington-Hall}\\
\pcmd{namequal}{Prof. H. Lington-Hall, CDM, DDT}\\
\pcmd{jobtitle}{Layabout}\\
\pcmd{extension}{6666}\\
\pcmd{loginid}{hlh}\\
\pcmd{emaildomain}{devnull.york.ac.uk}\\
\pcmd{website}{http://www.java.applets.R.us/}\\
\pcmd{definecolor}{brown$\}\{$rgb$\}\{$0.7,0.5,0.3}\\
\pcmd{headcolour}{brown}
\end{tabular}
}
\caption{An example sub-class of \cmdsty{UoYletter}}\label{Cls:Fig}
\end{figure}
\begin{figure}
{\sffamily
\begin{tabular}[t]{l}
\pcmd{documentclass[12pt]}{HLHletter}\\
\pcmd{date}{31 February 1944}\\[1ex]
\pcmd{begin}{document}\\
\pcmd{begin}{letter}\\
\hspace*{1cm}   $\{$Dr. A. Cademic\nl The House\nl Avenue Road\nl
                The City\nl CI9 0OO$\}$\\
\pcmd{ourref}{Sp/1/H-LH/GS}\\
\pcmd{yourref}{23 7-A/b}\\
\pcmd{opening}{Dear Arbuthnot,}\\
\pcmd{re}{Arachnophilia}\\
Thank you for your letter of 5th \pcmd{emph}{inst}.\\
The Tarantulas finished off Agatha.\\
I hope you enjoy the enclosed;\\
perhaps it could help with the\\
Problem Of The One Undergraduate Too Many!\\
\pcmd{closing}{Lurv `n' kisses,}\\
\pcmd{encl}{A Black Widow}\\
\pcmd{ps}{PS The Bird Eating Spiders don't.}\\
\pcmd{end}{letter}\\
\pcmd{end}{document}
\end{tabular}
}
\caption{A letter made with a private letter class}\label{ExampleInpTwo:Fig}
\end{figure}

See the Local Guide to find out how to persuade \LaTeX\ to take notice
of your personal letter class.

\section{Packages and files used by \cmdsty{UoYletter}}

Class \cmdsty{UoYletter} loads the standard package \cmdsty{letter},
and also the local style \cmdsty{UoYstyle}.  In turn \cmdsty{UoYstyle}
loads three standard styles: \cmdsty{palatcm} (for Palatino fonts),
\cmdsty{graphics} and \cmdsty{color}.

\bibliographystyle{agsm}
\bibliography{References} 
\end{document}
