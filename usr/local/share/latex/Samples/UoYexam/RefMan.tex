\documentclass[12pt,twoside]{article}
\usepackage{UoYstyle}
\title{Type-setting University Examinations\\in Computer Science\\using \LaTeX}
\author{Jeremy Jacob%
        \thanks{Extension 2747, e-mail $<$jeremy@minster.york.ac.uk$>$}}
\date{\fbox{DRAFT:~\today}}
\newcommand{\kwd}[1]{\textsf{#1}}
\newcommand{\cmd}[1]{$\backslash$\textsf{#1}}
\newcommand{\tcmd}[3]{\cmd{#1}$\{$#3{#2}$\}$}
\newcommand{\dcmd}[2]{\tcmd{#1}{#2}{\emph}}
\newcommand{\pcmd}[2]{\tcmd{#1}{#2}{\kwd}}
\begin{document}
\maketitle
\section{Introduction}
This document describes two \LaTeX2e\ document classes,
\kwd{UoYClosedExam} and \kwd{UoYOpenExam}, that support the
typesetting of University of York examination papers.  It assumes that
you are relatively familiar with \LaTeX\ in its latest incarnation
(\LaTeXe).  Its creator, Leslie Lamport, has written a user guide and
reference manual \cite{Lamport1994}; another useful book is by Kopka and
Daly \cite{KopkaDaly1995}.  You should also consult the
``local guide'' maintained by John Murdie \cite{LaTeXLocalGuide}.

The examination paper is set in the University style
\cite{UoYStyleGuide}.

The \kwd{UoYClosedExam} class also helps with the management of model
answers.  These are typed \emph{inline} with the questions, but appear
at the end of the paper, if at all (see Section \ref{ModelAnswer:S}).
Additionally the \kwd{UoYClosedExam} class reports to the log file the
difference between the number of marks alloted for a question and the
sum of the number of marks declared for each \kwd{part} within a
question (a positive number means not enough marks allocated in the
parts; a negative number means too many marks allocated in the parts).
This difference is not meaningful for questions with no sub-parts.

\section{Front matter}\label{Declarations:S}

The front matter in both classes is controlled by declarations that
appear in the preamble.  The declarations are listed in Table
\ref{Declarations:T}, together with the classes for which they are
appropriate.
\begin{table}
\begin{center}
\begin{tabular}[t]{l|l}
Declaration&Class\\\hline
\dcmd{Maccode}{MAC code}&Both\\
\dcmd{Examyear}{Year}&Both\\\hline
\dcmd{Title}{Title of paper}&Closed\\
\dcmd{Duration}{Time}&Closed\\
\dcmd{Rubric}{Text}&Closed\\
\cmd{Sectioned}&Closed\\\hline
\dcmd{Courses}{$\backslash\backslash$-separated list of courses}&Open\\
\dcmd{Degrees}{Degree list}&Open\\
\dcmd{Module}{Name}&Open\\
\dcmd{Part}{Examination part}&Open\\
\dcmd{Issue}{Time and date of issue}&Open\\
\dcmd{Due}{Time and date due back}&Open\\
\dcmd{Setter}{Name(s) of setters}&Open
\end{tabular}
\end{center}
\caption{Declarations for the front matter}\label{Declarations:T}
\end{table}
They are all compulsory (except for \cmd{Sectioned}), and \TeX\ will
halt with a suitable error message if they are ommitted.  The front
matter for a closed examination is printed on a page by itself; the
front matter for an open examination is printed on the top half of the
first page.

A degree list should be of the form:
\begin{verse}
        \pcmd{Degrees}{BSc Examination}\\
        \pcmd{Degrees}{BA and BEng Examinations}\\
        \pcmd{Degrees}{BSc, MEng and MSc Examinations}
\end{verse}

The part should be quoted as ``\pcmd{Part}{Ia}'', ``\pcmd{Part}{IIb}'',
and so on. 

``\cmd{Sectioned}'' is omitted if the paper is not split into
sections.  If it is included, then a list of the sections is included,
together with the rubric:
\begin{quote}
``Answer each Section in a \textbf{separate} answer booklet.''
\end{quote}
To get the list of sections, the file will have to be processed twice,
in the usual \LaTeX\ fashion for tables of contents, and so on.  (See
the ``\cmd{section}'' command, Section \ref{Section:S}.)

Durations should be quoted with their units: for example
\begin{quote}
``\pcmd{Duration}{three hours}''.
\end{quote}

The ``\pcmd{Rubric}{\ldots}'' declaration is for any further
miscellaneous instructions.  The argument may contain paragraphs
breaks.

\section{Sections}\label{Section:S}

Sections, if there are any, are introduced with the command:
\begin{quote}
        \dcmd{section}{name}$\{$\emph{rubric for section}$\}$
\end{quote}
The section label is generated automatically by \LaTeX, and can be the
target of a \pcmd{label}{\ldots} declaration.  If model answers are
being generated the section label, name and rubric are repeated with
the model answers (see Section \ref{ModelAnswer:S}).

The ``\cmd{Sectioned}'' declaration should be given if there are any
sections in the examination (see Section \ref{Declarations:S}).

The rubric may be empty, and may contain paragraph breaks.  It may
also contain the command ``\cmd{SeparateAnswerBook}'' which generates
the text:
\begin{quote}
        ``You must answer this section in a \textbf{separate} booklet.''
\end{quote}

\section{Questions}
Support is provided for two sorts of questions: ordinary, long
questions, and questions composed of a choice of short, independent
parts (the latter being a historical curiosity?).  The texts of the
two types of questions are typed in exactly the same way, although the
output is formatted slightly differently.

The question number can be the target of a \pcmd{label}{\ldots}
declaration.

If model answers are being produced, then, in both cases, the question
number, number of marks and rubric are printed with the model answers
(se Section \ref{ModelAnswer:S}).

\subsection{Ordinary questions}

Ordinary questions are typeset using an evironment ``\kwd{question}'':
\begin{quote}
        \pcmd{begin}{question}$\{$\emph{marks}$\}\{$\emph{rubric}$\}$\\
        \emph{text of question}\\
        \pcmd{end}{question}
\end{quote}
The number of marks should be given as just a number, for example:
\begin{quote}
``\pcmd{begin}{question}$\{$\kwd{20}$\}$\ldots''.
\end{quote}

The rubric may be empty, and may contain paragraph breaks.

Any sub-parts are numbered using lower case roman numerals.


\section{Texts of questions}
Question bodies can either be straight text (perhaps for an essay
question) or a series of parts.  Parts are numbered by lower case
roman numerals.

Each part has the form:
\begin{quote}
\cmd{part} \emph{text of part}
\end{quote}
or
\begin{quote}
\cmd{part[}\emph{mark for part}\kwd{]} \emph{text of part}
\end{quote}
If model answers are being produced, then the part number (and mark,
if given) are copied to the model answers (see Section
\ref{ModelAnswer:S}).  The \emph{mark for part} must be a number.

The part number can be the target of a \pcmd{label}{\ldots}
declaration.  Any normal \LaTeX\ commands can be included in
``\emph{text of part}''.

Two special environments for use within the \emph{text of part} are
\kwd{options} and \kwd{sequence}.  These are for a choice of optional
and sequence of compulsory subparts, respectively.  Each optional part
should be worth the same number of marks, and this should be stated in
the immediately preceeding rubric.  The format of of a series of
options is:
\begin{quote}
  \pcmd{begin}{options}\\
    \cmd{opt}
      \emph{text of optional part}\\
    \cmd{opt}
      \emph{text of optional part}\\
    \cmd{opt}
      \emph{text of optional part}\\
    \ldots\\
  \pcmd{end}{options}
\end{quote}
If model answers are being produced, then the option number is copied
to the model answers (see Section \ref{ModelAnswer:S}).  The option
number can be the target of a \pcmd{label}{\ldots} declaration.

The format of \kwd{sequence} is similar:
\begin{quote}
  \pcmd{begin}{sequence}\\
    \dcmd{bit}{mark for compulsory part}
      \emph{text of compulsory part}\\
    \dcmd{bit}{mark for compulsory part}
      \emph{text of compulsory part}\\
    \dcmd{bit}{mark for compulsory part}
      \emph{text of compulsoryl part}\\
    \ldots\\
  \pcmd{end}{sequence}
\end{quote}
If model answers are being produced, then the bit number and marks are
copied to the model answers (see Section \ref{ModelAnswer:S}).  The
bit number can be the target of a \pcmd{label}{\ldots} declaration.

\section{Model answers}\label{ModelAnswer:S}

The printing of model answers is controlled by a pair of class
options: \kwd{answers} and \kwd{noanswers}.  The default is
\kwd{noanswers}.  Option \kwd{answers} generates model answers for the
examination, and places them at the end of the text.  The model
answers contain all the Section, Question and Part headings, including
marks and rubrics.  The model answers also contain any comments placed
in the main body of text.  The purpose of this is to keep the source
text for the questions and answers tightly bound together.

Comments for inclusion in the model answers can be included in two
ways.  There is an ``\dcmd{ans}{text}'' declaration, for short
comments\footnote{``Short'' means that it must not include a paragraph
  break.} and an \kwd{answer} environment for long
comments\footnote{``Long'' means that it may include paragraph
  breaks.}  They may be used as follows:
\begin{quote}\sffamily
\ldots What is 2+3?\pcmd{ans}{5}
\end{quote}
\begin{quote}\sffamily
\ldots What is 2+3?\pcmd{ans}{2+3=5}
\end{quote}
\begin{quote}\sffamily
\ldots What is 2+3?\pcmd{ans}{A trick question, by convention 2+3=6 in
                                this module.}
\end{quote}
\begin{quote}
{\sffamily
\ldots Give the quicksort algorithm in Ada.}\\
\pcmd{begin}{answer}\\
\hspace*{2em}\emph{Ada code and other comments.}\\
\pcmd{end}{answer}\\
\end{quote}

Floats (constructs such the \LaTeX{} environments \kwd{figure} and
\kwd{table}) and footnotes are not supported in the model answers.

\section{Rough and final copy}
A pair of class options are provided to help version control.  The
default option is \kwd{final}, and this should be used for the final
production run of the paper.

For rough versions of the paper, the option \kwd{rough} should be
given.  This adds a message to the front of the paper, and to each
page's header declaring that it is a draft, and the date on which the
draft was processed by \LaTeX.

\section{Miscellaneous matters}
According to regulations, if a question is broken over an odd page the
foot should contain the text ``Continued.'', otherwise an odd foot
should contain the text ``Turn over.''.  The \kwd{UoYExamination}
class tries to manage this, but does not always succeed.  In cases
when it does not, one of the two following declarations should be
given to force the right message to appear.  The declaration should be
placed next to text which is known to appear on the page with the
incorrect message:
\begin{center}
\begin{tabular}{l|l}
Declaration&Message produced\\
\hline
\cmd{TurnOver}&Turn over.\\
\cmd{Continued}&Continued.
\end{tabular}
\end{center}
\bibliographystyle{plain}
\bibliography{References}
\end{document}
